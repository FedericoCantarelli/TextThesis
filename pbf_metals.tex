\setlength{\tabcolsep}{10pt}
In this Section, I will provide a comprehensive overview of powder bed fusion (PBF) processes in metal additive manufacturing. In Section \ref{sec:pbf_proc}, we will discuss how 3d printers for metal PBF work, in Section \ref{sec:metalpowders} we will discuss about the powders used in those processes, while in Section \ref{sec:matterint}, we will delve into the physical process that allows the selective fusion of metal powders into the finished product. Lastly, in Section \ref{sec:examplesPBF} we will discuss some applications for metal PBF processes.

% Powder Bed Fusion Processes >>>
\section{Powder Bed Fusion Processes}\label{sec:pbf_proc}
As we have already discussed in Section \ref{sec:AMproc}, powder bed fusion processes can be further divided into laser powder bed fusion processes, also called selective laser sintering (SLS), or electron beam powder bed fusion processes also called electron beam melting (EBM). Despite the fact that the two processes are based on two different technologies and some differences in printers components, the two processes are based on the same fundamental idea: selectively melting metallic powders using external energy sources. All PBF 3D printers consist of two basic components: the powder bed, which is a container that can move along z-axis, and a highly concentrated energy source. 

% SLS >>>
\subsection{Selective Laser Sintering}
\label{subsec:LPBF}
\begin{figure}
    \centering
    \includegraphics[scale=1.2]{Images/PBF.jpg}
    \caption[Laser PBF in AM.]{Laser powder bed fusion process in additive manufacturing. Adapted from \cite{ozel_focus_2020}.}
    \label{fig:PBF}
\end{figure}
Now we will examine the SLS printing process from start to finish. Firstly, the plate mounting is calibrated and a vacuum atmosphere is created inside the printer chamber or a regular flow of inert gas is introduced to obtain controlled and uniform melting of the metal powder, in order to minimize oxidation and degradation of the powdered material. In industrial context, directing an inert gas is preferred to maintaining a vacuum atmosphere. This mainly for two reason: maintaining a vacuum atmosphere throughout the entire printing process require a lot of efforts and lowering the pressure in the process chamber can lower the boiling point of the metals, leading to the formation of an excessive amount of smoke. The mixture of vaporized metal and smoke can affect the effectiveness of the laser. Indeed, laser is an optical device and, as we all know, Micro particles suspended in the air could deflect the laser beam. Once the inert gas flow is established, the roller or the blade responsible for distributing the powder performs a rapid powder re-coating. Then an electric resistor located under the powder bed preheats the powder at a temperature of about \SI{500}{\degreeCelsius}. This operation helps to reduce the gap between the temperature of the chamber and the metal's melting temperature and also helps in maintaining an uniform temperatures within the build tray. This is necessary in order to prevent the warping of the part during the build due to non-uniform thermal expansion and contraction, which will lead to cracks and fractures (see Section \ref{sec:defects}). After these preliminary operations, the printing process can begin. A scanning system composed of a laser diode and a galvanometric mirrors system selectively melts the metal, alternating with the powder distribution blade, layer after layer. LASER is an acronym that stands for "light amplification by stimulated emission of radiation," and it is a monochromatic beam of photons characterized by low divergence and a focal spot size of \numrange[range-phrase = --]{30}{80}\unit{\micro\metre}. Laser equipment has the capability to generate powers in the range of thousands of Watt and can focus to beam spot sizes of fractions of a millimeter. These small spot sizes have the potential to create minuscule molten pools that can melt at extremely high travel speeds, reaching up to several meters per second. Nowadays, almost all lasers used in AM rely on active optical fiber sources. A schematic representation can be seen in Fig. \ref{fig:detailedfiber}. 
\begin{figure}
    \centering
    \includegraphics[scale=0.45]{Images/galvanometro.png}
    \caption[Galvanometric system.]{Dual-axis galvanometric system.}
    \label{fig:galvano}
\end{figure}
The energy transferred from the laser to the powder bed depends on the laser power (\numrange[range-phrase = --]{100}{1000}\unit{\watt}), the light-absorbing capacity of the material, and the scanning speed, which can be controlled (and it's limited) by the angular velocity of the galvanometer. The galvanometer system, Fig. \ref{fig:galvano}, system consists primarily of two mirrors rotating along their axis, enabling the laser beam to be directed to a desired point. The photon beam passes through a series of f-theta lenses that facilitate rapid laser movement and precise focusing of the laser spot. After completing the printing process, it is necessary to allow the object to cool down. Although supports are not always necessary as the powder gives support to the object itself, they may need to be inserted to ensure uniform heat transmission during both the printing and cooling processes. Once the object has cooled down, any excess powder can be removed. The excess powder can be recycled and reused after undergoing some preliminary processing and conformity controls \cite{strondl_characterization_2015}. Finally, if required, post-processing operations can be carried out, such as removing any supports, surface finishing, and annealing. The latter is required if the cooling phase inside the chamber has created an undesired micro structure in the printed piece.
% <<<SLS

%%%%%
%%%%%
% EBM >>>
\subsection{Electron Beam Melting}
\label{subsec:ebm}
In EBM printers, the process is fundamentally the same as in laser printers, with metal powder selectively fused in a powder bed, with a recoater blade to provide additional metal powder layer after layer. However, the energy source is not a laser (i.e., a beam of photons) but rather an beam of electrons. Secondly, the preheating stage is not accomplished through the use of an electric resistor, but instead through the use of the electron beam itself. The preheating operation is one of the biggest difference between EBM and SLS. To produce an \emph{electron beam}, an electric current is passed through a metallic filament, typically tungsten or tantalum. The resulting electron beam is then directed through a Wehnelt cylinder. When the cylinder is negatively or positively charged, blocks or allows the passage of electrons respectively.
\begin{figure}
    \centering
    \includegraphics[scale=0.5]{Images/EBM.png}
    \caption[Electron gun schema.]{Electron gun schema.}
    \label{fig:electrongun}
\end{figure}
A diagram of the electron gun beam's structure can be seen in Fig. \ref{fig:electrongun}. Finally, the electron beam passes through an anode charged with a voltage of up to \SI{60}{\kilo\volt}, which accelerates the electrons and allows the beam to be directed into the column through magnetic lenses. Unlike SLS printers which rely on optical lenses, the lenses in EBM printers are magnetic lenses. A more comprehensive description of how magnetic lenses function will be provided in Section \ref{sssec:magneticlens}. EBM printers require a vacuum atmosphere inside the printing chamber to correctly work and a minute amount of helium must be continuously injected into the chamber to prevent the accumulation of electric charges due to any residual electrons on the powder bed. The phenomenon will be discussed in Section \ref{subsec:ebminter}. To put the "minuscule amount" of helium required into perspective, consider that the pressure in the vacuum atmosphere inside the chamber is \SI{0.05}{\pascal}, and the pressure with helium flow is about \SI{0,2}{\pascal} (recall atmospheric pressure is \SI{1,01325e5}{\pascal}). The second major difference, as mentioned earlier, is that the preheating phase does not occur through an electric resistor but rather through an initial scan of the electron beam that happens at two distinct moments. During the first phase, the entire powder bed is scanned, and during the second phase, only sub-regions of the powder bed that will actually be printed are scanned. These preheating phases are useful in avoiding the so-called "smoke effect," an effect caused mainly by the electrostatic repulsion in the powder. Furthermore, the preheating phase is necessary for obtaining final objects with better performance and reduces the need for support structures during printing. This allows for the use of the entire volume of the powder bed container when printing, and even printing multiple objects one on top of the other, since there would be no printing supports interfering with them. This dual preheating phase causes the excess powder to solidify around printed pieces. Another structural difference of an EBM printer can bee seen in Fig. ~\ref{fig:ebm_printer}. Due to the vacuum inside the print chamber and the consequent lowering of the metal's vaporization temperature, a heat shield is required around the chamber to prevent the vaporized metal from condensing and creating a film on the entire printer structure. Essentially, the heat shield acts as a physical barrier that allows for easy removal and disposal of the metal film created during the printing process.
\begin{figure}
    \centering
    \includegraphics[scale=0.3]{Images/A-schematic-of-electron-beam-melting-EBM.png}
    \caption[EBM 3d-printer.]{Electron beam melting 3d-printer \cite{azam_-depth_2018}.}
    \label{fig:ebm_printer}
\end{figure}
\begin{table}
\small
    \centering 
    \begin{tabular}{|l l l|}
    \hline
    \rowcolor{bluepoli!40} % comment this line to remove the color
     & \textbf{EBM} & \textbf{SLS} \T\B \\
    \hline \hline
    \textbf{Energy source} & Electron beam & Laser  \T\B \\ 
    \textbf{Chamber atmosphere} & Vacuum(Almost) & Inert gas  \T\B\\
    \textbf{Scanning} & Magnetic lenses & Galvanometers \T\B \\
    \textbf{Energy absorption} & Conductivity-limited & Absorptivity-limited \T\B\\
    \textbf{Scan speeds} & Very fast & Limited by galvanometer inertia \T\B\\
    \textbf{Energy costs} & Moderate & High \T\B\\
    \textbf{Feature resolution} & \numrange{100}{200}\unit{\micro\metre} & \numrange{75}{100}\unit{\micro\metre}  \T\B\\
    \textbf{Materials} & Conductors & Polymers, metals and ceramic \T\B\\
    \textbf{Layer Thickness} & \SI{50}{\micro\metre} & \SIrange{10}{50}{\micro\metre}  \T\B\\
    \textbf{Min Wall Thickness} & \SI{0.6}{\milli\metre} & \SI{0.2}{\milli\metre} \T\B\\
    \textbf{Accuracy} & $\pm$\SI{0.4}{\milli\metre} & $\pm$\SI{0.3}{\milli\metre} \T\B\\
    \textbf{Build rate} & \numrange[range-phrase=--]{55}{80}\unit{\centi\metre^3 / \hour} & \numrange[range-phrase=--]{60}{100}\unit{\centi\metre^3 / \hour} \T\B\\
    \textbf{Powder Particle Size} & \numrange[range-phrase = --]{40}{105}\unit{\micro\meter} & \numrange[range-phrase = --]{15}{45}\unit{\micro\meter} \T\B\\
    \hline
    \end{tabular}
    \\[10pt]
    \caption{Comparison between SLS and EBM processes. Adapted from \cite{gallina_electron_2017}.}
    \label{table:slsvsebm}
\end{table}
There are also differences in the materials used for printing. In EBM printing, usually metal powder size is slightly larger than the powder used in laser printing, with a particle size of approximately \numrange[range-phrase = --]{40}{105}\unit{\micro\meter} against the \numrange[range-phrase = --]{15}{45}\unit{\micro\meter} in SLS processes. Furthermore, metal powder must be made of metals that exhibit good electrical conductivity. Indeed, in EBM also highly reflective metals can be used, because unlike photons, electrons are not repelled by mirrored surfaces since they penetrate the matters thanks to the material's own electrical conductivity.

To sum up what has been discussed so far, Table ~\ref{table:slsvsebm} highlights the main differences between the two processes described in previous sections.
% <<< End of EBM
% <<< End of Powder Bed Fusion Processes

%%%%%
%%%%%

% Metal Powders >>>
\vfill
\section{Metal Powders for AM} 
\label{sec:metalpowders}
As we have seen in the section \ref{sec:pbf_proc}, PBF processes require metal powders. Various metallic materials can be transformed into powders suitable for use in additive manufacturing processes, and can be applied in different sectors according to their mechanical properties. Table \ref{table:materialAMmetal} presents some examples of metallic materials currently available and the respective industrial sectors in which they are used. In the vast majority of cases, these powders are produced using atomization processes that exploit various physical methods to generate micro-particles, characterized by a more or less spherical shape and of which chemical purity depends on the method used. The inefficiency and cost of atomization processes used for manufacturing metal powders is the reason why, in the case of specific high-quality powders, the powder price can be up to 10 times the price of raw metal. \citeauthor{deng_origin_2020} (2020) have demonstrated that powders made of irregular micro-particles and with low chemical purity can lead to structural defects in the final parts, resulting in lower performance. Therefore, over the years, increasingly complex and expensive processes have been developed to obtain higher-quality powders.
\begin{table}[H]
\centering 
\small
    \begin{tabular}{|l l l|}
    \hline
    \rowcolor{bluepoli!40}
    \textbf{Material} & \textbf{Examples} & \textbf{Applications}\\
    \hline \hline
    \textbf{Stainless steel} & 316L, 174-PH, MS1, M300 & Food, biomedical, consumer \T\B\\
    \textbf{Ni-alloys} & In625, In718, In939 & Energy, motorsport\T\B\\
    \textbf{Al-alloys} & AlSi12, AlSi10Mg & Lightweight, aerospace, aviation\T\B\\
    \textbf{CoCr-alloys} & CoCrMo & Dental, biomedical\T\B\\
    \textbf{Ti-alloys} & Ti6Al4V, CP Ti & Biomedical, lightweight, aerospace\T\B\\
    \textbf{Tool steel} & Maraging 18Ni300 & Tooling, aerospace, automotive\T\B\\
    \textbf{Cu-alloys} & Bronze & Energy, heat exchanger\T\B\\
    \textbf{Precious} & Au, Pt, Ag & Jewellery, design\T\B\\
    \hline
    \end{tabular}
    \\[10pt]
    \caption{Material availability for metal AM.}
    \label{table:materialAMmetal}
\end{table}
\paragraph{Water atomization.}The process of water atomization of metals, Fig. \ref{fig:wateratom}, involves the production of small droplets of molten metal by exposing a stream of molten metal to a high-pressure jet of water. The water atomization process is typically carried out in a chamber where the molten metal is injected into a nozzle that directs the stream of liquid metal towards a high-pressure jet of water. As the molten metal comes into contact with the water, it rapidly cools and solidifies, forming small droplets that are collected at the bottom of the chamber. The size and shape of the metal droplets produced during the water atomization process can be controlled by adjusting the temperature of the molten metal, the pressure of the water jet, and the distance between the nozzle and the water jet. By controlling these parameters, it is possible to produce metal powders with a range of particle sizes between \numrange[range-phrase = --]{1}{500} \unit{\micro\metre}. The output-to-input ratio for this process is approximately 95\%, which means that from \SI{1}{\kilo\gram} of raw metal, it is possible to obtain up to \SI{950}{\gram} of powder on average. Compared to other powder production methods, the water atomization process offers several advantages including high production rates, the wide range of particle sizes obtainable, and the fact that metal ingots can be used as process input, which shortens the manufacturing chain of this production process. On the other hand, the process is characterized by a low chemical purity, irregular shape as can be seen in Fig. \ref{fig:waterpow}, and needs extensive post-processing operations to obtain a decent-quality product. Moreover, it cannot be used with reactive materials such as titanium and aluminum.
\paragraph{Gas atomization.} The process of gas atomization, Fig. \ref{fig:gasatom}, can be performed using a variety of gases, depending on processed metals. The most commonly used gases are inert gases such as nitrogen, argon, and helium, even if the latter is barely used in industrial applications due to its high cost. Inert gases are preferred because they do not react with the molten metal and do not introduce impurities into the metal powders. Metal ingots are molten and the flow is rapidly solidified by exposing it to a high-pressure stream of gas. Main advantages of this process are the high production rate, the wide range of obtainable particles, its applicability to reactive materials, and its ability to ensure good chemical purity. The main drawbacks, are mostly related to the porosity of the resulting powder and the formation of small satellite particles.
\begin{figure}
    \centering
    \subfloat[\label{fig:wateratom}]{
        \includegraphics[scale=0.6]{Images/wateratom.png}
    }
    \qquad
    \subfloat[\label{fig:gasatom}]{
        \includegraphics[scale=0.6]{Images/gasatom.png}
    }
    \qquad
     \subfloat[\label{fig:plasmaatom}]{
        \includegraphics[scale=0.4]{Images/plasma.png}
    }
    \qquad
    \subfloat[\label{fig:repatom}]{
        \includegraphics[scale=0.7]{Images/repatom.png}
    }
    \caption[Atomization processes]{Water atomization process (a), gas atomization process (b), plasma atomization process (c), rotating electrode process (d) \cite{material_technology_innovations_co_rotating_nodate, material_technology_innovations_co_water_nodate, material_technology_innovations_co_gas_nodate, inovar_communications_ltd_metal_2020}.}
    \label{fig:atom}
\end{figure}
\paragraph{Plasma atomization.} The process of plasma atomization, Fig. \ref{fig:wateratom}, involves the usage of high-temperature plasma torches to melt metal wire. The high-energy plasma arc is created by passing an electric current through a non-consumable tungsten electrode and an inhert gas jet to direct the welding arc into a focused area. As the molten metal droplets are expelled from the plasma jet, they rapidly solidify, sometimes also thanks to the water-cooled chamber, and break up into small, spherical powders, which are then collected in a container. The plasma atomization process offers several advantages over other powder production methods, including the ability to produce extremely spherical shape. Moreover, it can be used with reactive material. Plasma atomization can also produce powders with very high purity levels, as the high temperature of the plasma jet helps to eliminate impurities in the molten metal. However, it is more expensive than other powder production methods due to the need for specialized equipment and the high energy consumption required to generate the plasma arc. The process can also be challenging to control, as variations in parameters can affect the size, shape, and purity of the resulting powders. Powders produced using this method has a low size range of about \numrange[range-phrase = --]{1}{200}\unit{\micro\metre}.
\paragraph{Rotating electrode process.} In rotating electrode process, Fig. \ref{fig:repatom}, a consumable metal electrode is rotated at high speeds while it is melted by an electric arc, in a chamber with a continuous inert gas flow. As the molten metal is exposed to the high-velocity inert gas stream, it rapidly solidifies and breaks up into small droplets, which are then collected at the bottom of the atomization chamber. Due to its high cost, this production method is typically used when it is necessary to obtain a metal powder with a perfectly spherical shape and without any impurities, ad example gold. If we take a look at Fig. \ref{fig:reppow}, we can easily grasp the potential of this technology in obtaining perfect spherical micro-particles.
\begin{figure}
    \centering
    \subfloat[\label{fig:waterpow}]{
        \includegraphics[scale=0.22]{Images/waterpow.png}
    }
    \qquad
    \subfloat[\label{fig:gaspow}]{
        \includegraphics[scale=0.21]{Images/gaspowd.png}
    }
    \qquad
     \subfloat[\label{fig:plasmapow}]{
        \includegraphics[scale=0.22]{Images/plasmapowder.png}
    }
    \qquad
    \subfloat[\label{fig:reppow}]{
        \includegraphics[scale=0.21]{Images/erppowder.png}
    }
    
    \caption[Powders from atomization processes]{Micro-particles obtained by water atomization (a), gas atomization (b), plasma atomization (c) and rotating electode process (d) \cite{slotwinski_characterization_2014}.}
    \label{fig:powders}
\end{figure}
% <<< End of Materials

%%%%%
%%%%%

% Energy matter interactions >>>
\section{Energy - Matter Interactions}
\label{sec:matterint}
This section should be regarded as "optional", and it's up to the reader's discretion whether to engage with it or not. In this section, we will delve deeply into the physical processes underlying the technologies discussed in Section \ref{sec:pbf_proc}. As I have already said, in SLS there is the interaction between a photon and metal powder, while in EBM there is the interaction between electrons and metal powder. I will discuss first laser-metal interactions and then electrons-matter interaction.
\subsection{Laser - Matter Interaction}
\label{subsec:sintering}
The type and color of the laser are two extremely important features in SLS. Indeed, the energy transmitted to the material depends on energy absorption characteristics of the material itself that vary according to wavelength of the laser source. In order for the fusion process to be successful, metallic powder bed particles must receive enough energy to melt on the atomic level.
\begin{figure}
    \centering
    \subfloat[\label{fig:laserintera}]{
        \includegraphics[scale=0.4]{Images/laser interactions.png}
    }
    \qquad
    \subfloat[\label{fig:laserintensity}]{
        \includegraphics[scale=0.35]{Images/laserintensity.png}
    }
    
    \caption[Laser interactions and laser intensity.]{Schematic representation of physical interaction of absorption, reflection, transmission, refraction, and scattering between laser beam and material (a) \cite{katayama_fundamentals_2020}, and laser intensity schema (b).}
\end{figure}
For example, we can see that in Fig. \ref{fig:ondette} that absorptivity of copper is about 50\% and 58\% for green and blue lasers, respectively. In other words, green and blue lasers are more advantageous in processing copper sheets in terms of higher initial absorption than other laser typologies. As we will explore later, the electron beam transfers energy to the material through charge movement, while the laser beam operates as an optical device and hence it is more sensitive to the behaviors illustrated in Fig. \ref{fig:laserintera}. Indeed, the metal specimen is highly reflective and opaque. This is also the reason why the electron beam penetrates deeper into the metallic powder compared to the laser beam. 
\begin{figure}
    \centering
    \includegraphics[scale=0.4]{Images/ondette.png}
    \caption[Material absorptivity as function of wavelength.]{Absorptivity of incident light on several materials at room temperature as function of wavelength \cite{katayama_fundamentals_2020}.}
    \label{fig:ondette}
\end{figure}
The intensity of the laser emission varies as a function of the radius as 
\begin{equation}
    \label{eq:intensitylaser}
    I(r)=I_0\cdot e^{-2 \frac{r^2}{r_0^2}}
\end{equation}
and a schematic representation of density is shown in Fig. \ref{fig:laserintensity}.
The energy density of the laser beam is described by the equation
\begin{equation}
    \label{eq:energydensity}
    E_d = \frac{P}{h\cdot v \cdot d}
\end{equation}
where $E_d$ is the energy density (\unit{\joule/\milli\metre^3}), $P$ is the laser power (\unit{\watt}), $v$ is the laser scan speed (\unit{\milli\metre / \second}), $h$ is the hatch distance (\unit{\milli\metre}), $d$ is the layer thickness of the powder(\unit{\milli\metre}). For a more detailed description of heat transfer, please refer to next section. The intensity of the laser beam's energy plays a crucial role in SLS because if it is not correctly calibrated, the final piece might exhibit porosities that would compromise its mechanical features. Another critical parameter to consider during the SLS process is the duration of the laser pulse. As depicted in the Fig. \ref{fig:pulsipulsiocomepulsi} a prolonged laser pulse results in a broad heat zone, causing a shock wave that leads to an increased debris formation during the interaction that can interfere with new layers.
\begin{figure}
    \centering
    \includegraphics[scale=0.35]{Images/laserpulsing.png}
    \caption[Material absorptivity as function of wavelength.]{Schematic of laser interaction with materials under different pulse duration: (a) long pulse duration and (b) short pulse duration and respective holes fabricated on a steel foil by (c-d) \cite{lin_femtosecond_2021}.}
    \label{fig:pulsipulsiocomepulsi}
\end{figure}
For this reason, in recent years, there has been a shift towards using ultra-short pulse lasers. A schematic representation of one of this laser can be seen in Fig. \ref{fig:duripoco}. The laser in the figure is called "femtosecond laser" and is employed in manufacturing contexts such as precision processing or correction of semiconductors and liquid crystals, processing of transparent materials like glasses and sapphires, production of waveguide tubes for optical communication, drilling of engine components for cars and aircraft, among other applications \cite{katayama_fundamentals_2020}.
\begin{figure}
    \centering
    \includegraphics[scale=0.4]{Images/duripoco.png}
    \caption[Ultrashort pulse laser.]{Emission mechanism of ultrashort pulse laser \cite{katayama_fundamentals_2020}.}
    \label{fig:duripoco}
\end{figure}
\subsubsection{Heat transfer}
\label{sssec:heattransfer}
The heat and mass related events in SLS are influenced by both heat and mass movement. Laser heating is extremely rapid due to the fast scanning laser velocities. In a closed system, with respect to the first law of thermodynamics, the energy balance equation is formulated as follows \cite{bouabbou_understanding_2022}:
\begin{equation}
    \label{eq:tutteQ}
    Q_L = Q_C + Q_{CV} + Q_R,
\end{equation}
where $Q_L$, $Q_C$, $Q_{CV}$ and $Q_R$ respectively are the laser heat flux quantity, conduction, convection and radiation heat quantities. It is important to recall that only $Q_C$ will contribute to the melting process. Thus, some of the heat applied on the powder bed will be lost in a form of heat convection and radiation. To describe the heat conduction, we can use Fourier's law:
\begin{equation}
    \label{eq:nonsiecapitouncazzo}
    \frac{\delta}{\delta x}\left(k \frac{\delta T}{\delta x}\right)+\frac{\delta}{\delta y}\left(k \frac{\delta T}{\delta y}\right)+\frac{\delta}{\delta z}\left(k \frac{\delta T}{\delta z}\right)+\dot{q} =\rho C_p \frac{\delta T}{\delta t}
\end{equation}
where $T$ (\unit{\kelvin}) is the temperature, $k$ (\unit{\watt.\metre^{-1}.\kelvin^{-1}}) is the thermal conductivity, $\dot{q}$ (\unit{\watt.\metre^{-3}}) is the rate of which the heat is applied to the system, $r$ (\unit{\kilo\gram.\metre^{-3}}) is the material density, $C_p$ (\unit{\joule. \kilo\gram^{-1}.\kelvin^{-1}}) is the specific heat and $t$ (\unit{\second}) is the interaction time between laser beam and powder particles.

\subsubsection{Optical Fiber Laser}
\label{sssec:fiberlaser}
One of the most common types of lasers used in SLS is the fiber optic laser \cite{milewski_additive_2017}. A diode is a semiconductor component which basically acts like an unidirectional gate for electric current. It permits current to pass effortlessly in one direction but significantly hinders its flow in the opposite one. Diodes have a polarity, characterized by its anode (positive terminal) and cathode (negative terminal). Typically, diodes conduct current only when a positive voltage is supplied to the anode. In a laser diode (LD), an electric current is channeled directly through a dual hetero-junction structured semiconductor from the external side. When electrons recombine with positive holes at the active layer located between the n-type and p-type semiconductor zones, as shown in Fig. \ref{fig:np}, there is the emission of light beam.
\begin{figure}
    \centering
    \subfloat[\label{fig:np}]{
        \includegraphics[scale=0.35]{Images/np.png}}
    \qquad
    \subfloat[\label{fig:detailedfiber}]{
        \includegraphics[scale=0.35]{Images/detailedfiber.png}}
    \caption[Laser diode and fiber laser.]{Emission mechanism of laser diode (a) and a detailed fiber laser schema (b) \cite{katayama_fundamentals_2020}.}
    
\end{figure}
In fiber lasers, the beam is emitted by an LD pumping and a fiber of high purity \ce{SiO2} quartz glass doped with \ce{Yb^3+} rare earth element. The diameter of the fiber is about \numrange[range-phrase=--]{10}{20}\unit{\micro\metre}. In this laser technology, optical pump diodes are coupled to an active laser fiber that has a unique reflective coating and Bragg gratings, that reflect the laser light back-and-forth, along the length of the fiber, to create a coherent beam of light at the output of the laser \cite{milewski_additive_2017}. In Fig. \ref{fig:detailedfiber} we can see full reflection mirror and partial reflection mirror installed next to the fiber. This means also that the adjustment of the laser beam is not needed, which means easier handling of the system. It is also possible to combine multiple laser modules using a beam combiner for more powerful laser. We can use additional optical fibers to deliver and contain the light energy, providing a robust, flexible, and fully enclosed beam path for beam delivery \cite{milewski_additive_2017}. Fiber laser has many \textit{advantages} such as high beam quality, the fact they are small and lightweight, high intensity (which allows higher scan speed) and high efficiency (cost decreasing), long-distance delivery, and require basically no maintenance. 
\subsection{Electrons-Matter Interaction}
\label{subsec:ebminter}
According to \citeauthor{schultz_h_electron_1994} (1994), while the physical characterization of electron beams has been recognized since the 1960s, the microscopic processes are such that they remain without a comprehensive quantitative explanation also nowadays. However, a comprehensive explanation of electrons properties and their interaction with matter was given by \citeauthor{krumeich_properties_2015} (2015).
\subsubsection{What is an Electron?}
\label{sssec:electron}
An electron is a fundamental subatomic particle that carries a negative electric charge and it's one of the primary constituents of atoms. As explained in Section \ref{subsec:ebm} the electron beam is made of electrons generated through the thermionic effect with a metallic filament, typically tungsten or tantalum. At the cathode, an accelerating voltage $V$ is applied, causing electrons acceleration till velocity $v$. As explained in \citeauthor{krumeich_properties_2015} (2015), we can write the equation for kinetic energy associated to an electron as 
\begin{equation}
    \label{eq:energyelectron}
    E  = e\cdot V = \frac{1}{2}mv^2
\end{equation}
Given the fact that electron mass is \SI{9.109e-31}{\kilo\gram}, electron charge is \SI{-1.602e-19}{\coulomb} and that a common voltage for EBM application is \SI{60}{\kilo\volt}, from \ref{eq:energyelectron} we find that electron velocity is
\begin{equation}
\label{eq:velocityelectron}
v=\frac{\sqrt{2\cdot e\cdot V}}{m}
\end{equation}
which is approximately \num{1.453e8} \unit{m/s} (about half the speed of light). Recalling the dualism wave-particle of the electron, we can use De Broglie equation to compute the wavelength of the electron beam \cite{krumeich_properties_2015}:
\begin{equation}
\label{eq:wavelength}
    \lambda = \frac{h}{\sqrt{2\cdot m \cdot e \cdot V}}
\end{equation}
where $h=$\SI{6.62607015e-34}{\joule.\second} is Planck constant. In case of a voltage higher than \SI{300}{\kilo\volt}, it's necessary to include a term that accounts for the relativistic nature of mass in \ref{eq:wavelength}, as the electron's speed would approach the speed of light.
\subsubsection{Magnetic Lenses in EBM}
\label{sssec:magneticlens}
In EBM, electron beam is controlled by electromagnetic lenses. As the beam passes through these lenses, the astigmatic lenses adjust the shape of the electron beam spot, the focus coil changes the spot size, and the deflection coil moves the beam along the x and y axes.
\begin{figure}
    \centering
    \includegraphics[scale=0.9]{Images/Magnetic_lens.jpg}
    \caption[Magnetic lens.]{Forces schema when an electron beam pass through magnetic lens \cite{wikipedia_magnetic_2023}.}
    \label{fig:magneticlens}
\end{figure}
The force $\mathbf{F}$ which an electron of charge $-e$ experiences when traveling with a velocity $\mathbf{v}$ due to a magnetic field $\mathbf{B}$ is given by Lorentz's law:
\begin{equation}
    \mathbf{F} = -e\cdot \left( \mathbf{v} \times \mathbf{B}\right)
\end{equation}
We can decompose magnetic field into a radial component $\mathbf{B}_R$ and into an axial component $\mathbf{B}_L$. The initial direction of the electron is parallel to the axis, thus it is subjected to the radial component $\mathbf{B}_R$ only, which is responsible for an orthogonal force to $\mathbf{v}$. Hence, electrons start rotating. When they start rotating, they are subject to the axial field of the magnetic field too, which result in a reducing radius spiraling trajectory that converge the beam towards the target.
\subsubsection{Electrons Interaction}
\label{sssec:electroninteractions}
As explained in \citeauthor{korner_additive_2016} (2016), we can distinguish two interaction types between electrons and matters, defined on the basis of transferred energy quantity. 
\paragraph{Elastic interactions} The energy of the electron remains constant: when electrons delve into the surface, they engage as negatively charged entities with the metal's negative field and the positive charge from the nucleus's protons. Electrons can be rerouted on a different path without losing kinetic energy (known as elastic scattering). After several interactions, these deflected electrons might sometimes be ejected from the metal. This phenomenon is called referred to as backscattering. In Fig. \ref{fig:electronsscattering} there are a representation of all the possible interactions between an electron and an atom nucleus. This event occurs because of the Coulomb interaction, which happens when a negatively charged entity (electron) comes close to a positively charged one (nucleus). Coulomb force is described by the equation:
\begin{equation}
    \label{eq:coulomb}
    F=\frac{Q_1\cdot Q_2}{4\pi \varepsilon_0 r^2}
\end{equation}
where $r$ is the distance between the charges $Q_1$ and $Q_2$ and $\varepsilon_0=$ \SI{8.854e-12}{\farad / \metre} is the dielectric constant. 
\begin{figure}
    \centering
    \includegraphics[scale=0.4]{Images/electrons.png}
    \caption[Scattering and backscattering of an electron.]{Scattering and backscattering effect of an incident electron inside the electron cloud of an atom \cite{krumeich_properties_2015}.}
    \label{fig:electronsscattering}
\end{figure}
The backscatter phenomenon is determined by material characteristics and by the angle at which the beam hits the metal surface, see Fig. \ref{fig:backscattering}. It represents an energy deduction from the melting procedure. Indeed, if an electron is bounced outside the material, it cannot transfer energy to the material itself. So, for process efficiency, those interactions should be avoided. Backscattering effect is directly proportional to the atomic number $Z$ of the material of the powder, specifically as $Z^2$.
\begin{figure}
    \centering
    \includegraphics[scale=0.4]{Images/backscattering.png}
    \caption[Backscattering of an electron at different angles.]{Montecarlo simulation for an aluminum specimen hit by an electron beam at different angles, \ang{0}, \ang{45}, \ang{60}, \ang{75} respectively \cite{goldstein_scanning_2018}. In red, backscattered electrons.}
    \label{fig:backscattering}
\end{figure}
Having a fine and as spherical as possible powder is important to reduce backscattering effect caused by incidence angle. In EBM processes, the more spherical the particles are, the more we can assume that the electron beam will strike the particle surface perpendicularly. This can be clearly seen in Fig. \ref{fig:rotondette}.
\paragraph{Inelastic Interactions:} In inelastic interactions, the energy of the electrons diminishes. The energy imparted to the sample results in the heating of the metal and its correspondingly melting. These interactions causes the release of certain emissions, detailed below \cite{krumeich_properties_2015, goldstein_scanning_2018}:
\begin{itemize}
    \item \textbf{Inner-shell ionisation:} The energy is passed to an inner electron, which is adequate to eject it from the atom. This results in an unstable atom configuration, leading to the release of X-rays or of an Auger electron. \emph{X-rays} are produced when an inner electron is released into the vacuum, and an electron from the valence shell descends to occupy the void left by the electron from a deeper level. On the other hand, an \emph{Auger electron} is produced when the core electron, stimulated by the electron beam's energy, is expelled into the vacuum. The left space is occupied by another electron from a higher level, which then radiates energy to closer electron until it is released into the chamber. That's the reason why in EBM 3d printer is crucial to have a shield to protect operators from X-rays.
    \item \textbf{Braking radiation:} The electron that enters the atom slows down due to its interaction with the positively charged nucleus. The subsequent reduction in energy is released as X-rays. X-rays of lower energy are completely absorbed by the material, while X-rays of higher energy are released in the printing chamber.
    \item \textbf{Secondary electron:} Electrons within the valence band require minimal energy to surpass the potential barrier, and various electron-matter interaction mechanisms can provide this energy. This can cause the release of this electrons in the printing chamber. The flow of inhert gas in the chamber is used to remove these electrons from the powder bed, since they can interfere with printing process.
    \item \textbf{Cathodoluminescence:} when an electron from the valence band rises to the conduction band, a vacancy is created in the valence band. This vacancy is then occupied by another electron descending from the conduction band, which subsequently releases a photon. This phenomenon is responsible for the characteristic light produced when the electron beam hits the metal powder in EBM. 
    \item \textbf{Plasmon:} an electron moving through the collection of electrons in the valence band can cause a disturbance, leading to a collective vibration of the free electrons. This interaction is prevalent in metals but can also take place in any substance with free electrons. This phenomenon is responsible for the creation of plasma oscillation.
    \item \textbf{Phonos:} Phonons are understood as the collective oscillations of atoms within a crystal lattice, resulting from inelastic engagements with the electron beam. When an atom initiates this vibrational movement, it conveys this activity throughout the lattice, impacting a significant volume. Phonos are the only responsible for the melting of metal powders. Other products are undesirable, or even accountable for side effects during the printing process.
\end{itemize}
\begin{figure}
    \centering
    \subfloat[\label{fig:rotondette}]{
        \includegraphics[scale=0.7]{Images/EBMparticles.png}
    }
    \qquad
    \subfloat[\label{fig:unsaccodiroba}]{
        \includegraphics[scale=0.4]{Images/Screenshot 2023-08-21 at 11.42.27.png}
    }
    \caption[Laser interactions and laser intensity.]{Electron beam interaction with spherical powder particles (a) and Product from the interaction between an electron and a metal specimen (b) \cite{tushar_ramkrishna_mahale_electron_2009, krumeich_properties_2015}.}
\end{figure}
In Fig. \ref{fig:unsaccodiroba}, there is a schematic representation of the products resulting from the collision between an electron and a metal specimen.
% <<< End of Energy Matter Interactions

%%%%%
%%%%%

% Applications of PBF Metal Additive Manufacturing >>>
\section{Applications of PBF Metal Additive Manufacturing}
\label{sec:examplesPBF}
One of the primary advantages of PBF processes is the capability to produce finished objects unattainable with traditional manufacturing systems. In this section, we'll explore some examples of how this technology can completely disrupts production processes of metal manufacturing. The nozzle in Fig \ref{fig:nozzle} is one of the more famous successful and documented examples of additive manufacturing for metal. It was designed by General Electrics in 2015. With AM, GE was able to produce the new nozzle in a single piece, rather than 20 separated pieces welded together. The weight was reduced by 25\%, it was five times more durable and 30\% more cost-efficient \cite{amy_kover_transformation_2018}. GE Aviation has set the goal of 32,000 nozzles per year when in full production \cite{milewski_additive_2017}, opening this manufacturing process to large scale distribution. In 2015, NASA designed the first 3D-printed rocket nozzle made of copper. Within the combustion chamber, the propellant burns at temperatures exceeding \SI{3000}{\kelvin}. To prevent it from melting, hydrogen, at temperatures just above \SI{100}{\kelvin} flows through over 200 intricately designed cooling channels. Cooling inlets can be seen along the chamber's top rim. Thanks to this new rocket nozzle, costs were reduced by 50\%, and manufacturing times were reduced by ten times \cite{tracy_mcmahan_nasa_2015}, opening the path to a more affordable and lean space industry. An application of PBF processes in the automotive sector is illustrated in Fig. \ref{fig:piston} by Porsche. The company managed to print pistons by PBF for 911 GT2 RS engine. These new pistons have achieved a temperature on the piston's o-ring that is \SI{20}{\degreeCelsius} lower than usual. This lead to an additional \SI{23}{\kilo\watt} power available. However, before we see large-scale production, we'll have to wait until at least 2030 \cite{roberto_baldwin_porsches_2020} according to Porsche's press release. AM is also being employed in the medical field, for example, in dentistry. The traditional ways of fabricating items like crowns and dental implants are being transformed by custom-fitted designs. Advanced high-precision 3D printers such as EOS M 100 DMLS are used to craft devices from Cobalt Chrome SP2 alloy, a medical-grade approved material to be used in medical field \cite{milewski_additive_2017}. Products with a small batch size, high precision, and significant value, like these dental devices, are increasingly being manufactured using these new technologies. The advantages of AM include the swift creation of tailor-made items for immediate application, such as implants, or indirect uses like drill guides and fixtures, all based on the patient's own medical characteristics for a perfect fit. Moreover, the surface finish or porous structures offer huge advantages for bone growth. In Fig. \ref{fig:skul} there is an example of a titanium skull implant.
\begin{figure}
    \centering
    
    \subfloat[\label{fig:nozzle}]{
        \includegraphics[scale=0.20]{Images/nozzle.png}
    }
    \qquad
     \subfloat[\label{fig:nozzlerocket}]{
        \includegraphics[scale=0.24]{Images/nozzlerocket.png}
    }
    \\
    \subfloat[\label{fig:piston}]{
        \includegraphics[scale=0.21]{Images/piston.png}
    }
    \qquad
    \subfloat[\label{fig:skul}]{
        \includegraphics[scale=0.20]{Images/skull.png}
    }
    
    \caption[Examples of AM in metals.]{Examples of AM metal part: GE nozzle used in airplane turbines to mix fuel and air (a), a rocket nozzle in copper by NASA (b), Porsche engine pistons (c) and a titanium skull implant (d).}
    \label{fig:amexamples}
\end{figure}
\subsection{Lattice Structure and Cellular Material} \label{subsec:lattice}
A specific application of PBF processes is the printing of so-called lattice structures. According to ISO/ASTM 52900 \cite{international_standard_organization_isoastm_2015}, \emph{lattice structure} are "three dimensional geometric arrangement composed of connective links between vertices (points) creating a functional structure". Lattice structures are three-dimensional structures made up of different connected single elements called "cells" that can have different shapes, designed to meet the desired mechanical features of the final object. These objects are distinguished by empty spaces within the structure. These structures can only be obtained thanks to layer-wise AM technologies. In recent years, biomimicry technique has also spread in additive manufacturing. Biomimicry is a set of design techniques that allow us to take inspiration from nature to find solutions to engineering problems or to design functional components that can be used in engineering applications \cite{pathak_biomimicry_2019, du_plessis_beautiful_2019}. Nature, over the past 3.8 billions years, has been able to find the most efficient way to develop functional solutions to evolutionary problems characterized by immovable constraints, whether they are determined by the organism itself or by external factors. Moreover, during the evolutionary process of species, nature has been able to create nano, micro, and macro-structures that provide unique structural properties, adapting shape to function and using only what is truly necessary to achieve the evolutionary goal. Indeed, one of life's principles explained in \citeauthor{baumeister_biomimicry_2011} (2011) states \textit{"Life integrates and optimizes these strategies to create conditions conducive to life"}. Therefore, it not only concerns the possibility of creating new multi-functional structures with completely new characteristics, but also of doing so by optimizing materials and reducing waste. Engineers have always been fascinated by cellular materials, which is evident from the fact that the first reference to the idea that structure could influence the functional characteristics and behavior of a material was made by Robert Hooke in 1665 \cite{l_gibson_cellular_2010}. Only in recent years, thanks to the computational power of CAD softwares, it has been possible to experiment with the mechanical characteristics of 3D printed lattice structures. Cellular materials are used mainly for their mechanical characteristics.
\begin{figure}
    \centering
    \subfloat[\label{fig:beehive}]{
        \includegraphics[scale=0.15]{Images/honey-bees-337695_1280.jpg}
    }
    \quad
    \subfloat[\label{fig:beelattice}]{
        \includegraphics[scale=0.186]{Images/lattice1b.jpg}
    }
    \caption[Bio-inspiration design.]{An example of bio-inspiration design: from a beehive (a) to a lattice structure (b).}
    \label{fig:bioinsp}
\end{figure}
Just think that the aluminum cube in Fig. \ref{fig:beelattice} was printed using only \SI{3.9}{g} of material and is able to support a weight of \SI{408}{Kg}, which means it can withstand a stress 100,000 times its weight \cite{noauthor_3d_2014}. If we want to provide a more rigorous framework for classifying the applications of lattice structures in engineering, we can refer to the one proposed by \citeauthor{mcnulty_framework_2017}. According to this framework, cellular structures in nature exist primarily for three reasons:
\begin{itemize}
    \item \textbf{3D space-filling structures}, maybe this is the reason that most closely links the use of cellular structures in nature with AM, namely the need to confer strength to the structure while limiting its weight. Examples of this are the beehive, whose structure provides it with High specific stiffness under self-weight;
    \item \textbf{Surface structures}, enable surfaces to gain specific functional characteristics. For example, veining on the underside of the Amazon water lily leaf allows the leaf to gain excellent mechanical resistance, or the pomelo skin. with its open-cell structure, enables the fruit to resist impacts;
    \item \textbf{Cylindrical structures}: there are also examples of the use of cellular materials to confer particular characteristics to cylindrical structures (often hollow structure) that would otherwise be too fragile. For example hedgehog quills have a particular internal structure that confers ovalization and buckling resistance to the quill.
\end{itemize}
Lattice structures can be used for all these purposes, highlighting the wonderful connection between nature and engineering. Lattice structures are versatile and can be effectively applied in various engineering fields to address multiple challenges. Specifically, they are commonly used in four main areas according to \citeauthor{bhate_classification_2019} (2019): 
\begin{itemize}
    \item \textbf{Structural engineering} for vibration control, strain isolation, and weight reduction purposes like in Fig. \ref{fig:f1freno};
    \item \textbf{Thermal engineering} for applications such as heat exchangers in Fig. \ref{fig:heatexchanger}, flame arresters, or heat shields;
    \item \textbf{Fluid dynamics engineering} which employs lattice structures as catalyst carriers or packaging;
    \item \textbf{Biomedical engineering}, where lattice structure can be leveraged for bone integration in prosthetic and cell growth
\end{itemize} 
Due to their unique properties, lattice structures are an attractive choice for engineers and designers seeking innovative solutions to complex problems across a wide range of industries, especially in the automotive, aerospace, energy and commodities, and medical industries.
\begin{figure}
    \centering
    \subfloat[\label{fig:heatexchanger}]{
        \includegraphics[scale=0.25]{Images/heatexchanger.png}
    }
    \qquad
    \subfloat[\label{fig:f1freno}]{
        \includegraphics[scale=0.3]{Images/f1freno.png}
    }
    \caption[Lattice structure applications.]{Some lattice structure application: a complex heat exchanger (a) and brake pedal of F1 racing car (b) \cite{milewski_additive_2017, du_plessis_beautiful_2019}}.
\end{figure}
% <<< End of Applications of PBF Metal Additive Manufacturing