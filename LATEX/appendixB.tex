This appendix suggests a potential implementation of the Bagging Voronoi classificator algorithm discussed in Chapter \ref{ch:baggingvoronoi}. The algorithm was implemented in Python because it is currently the most widely used general-purpose programming language among developers \cite{lionel_sujay_vailshery_most_2023}. Furthermore, being a high-level and straightforward programming language, Python is easily comprehensible even to those unfamiliar with coding. The same algorithm can be implemented in C or C++ for real-time defect detection. Indeed, the same program written in C++ can be up to 20 times faster than the one written in Python \cite{zehra_comparative_2020,towards_data_science_how_2020}. In addition, because of the low-level process management in those programming languages, it is possible to use concurrent programming to parallelize the execution of the algorithm to take advantage of all available cores. The bootstrapping nature of the algorithm makes it particularly suitable for implementation by exploiting parallelization logic \cite{secchi_bagging_2013}. K-means clustering was used to perform the clustering stage using the Euclidean distance, and clustering matching method was done by using principal diagonal permutations of the contingency matrix. A kernel with a bandwidth of 1.5 was used for data smoothing, and the first $p$ PCs retained were those that explain 95\% of the total variance, as explained in Section \ref{sec:irradiance}.
\section{Python Code}
\lstinputlisting[language=Python]{algorithm_to_latex.py}
