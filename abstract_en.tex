% ABSTRACT IN ENGLISH
Metal Additive Manufacturing (AM) techniques, often called 3D printing, introduce innovative production approaches for crafting metal components through incremental material deposition, layer after layer. These technologies allow the production of complex shapes, lightweight structures, internal features, and surface patterns that can not be achieved with traditional methods. Recent technology development has accelerated the industrial adoption of metal AM procedures, particularly  Electron Beam Melting and Selective Laser Sintering. These two metal AM methods leverage an external concentrated energy source to melt metal powder selectively.
This thesis undertakes a comprehensive appraisal of PBF technology, providing an in-depth evaluation of operational principles, challenges, benefits, and critical issues compared with other processes, the typical defects, and approaches to defect detection and control quality. The thesis emphasizes detecting "hot spot" defects, localized zones exhibiting elevated temperatures and slower cooling drift, often attributable to their proximity to adjacent loose powder. In the context of metal part AM, these defects assume particular significance as most anomalies in these processes stem from anomalous thermal behaviors. A component's thermal history significantly influences the finished product's critical aspects, including porosity, microstructure, and geometry. Thus, it is essential to correctly understand and analyze the thermal history of the part, especially during the test and certification phase. The thesis will also present an algorithm for clustering areas on the build plate that share analogous thermal histories through functional data analysis. The thesis will also show an implementation of the algorithm in Python.
\\[0.5cm]
\textbf{Keywords:} metal additive manufacturing, powder bed fusion, defect detection, hot-spot, machine learning, functional data analysis
