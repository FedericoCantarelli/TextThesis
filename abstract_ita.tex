% ABSTRACT IN ITALIAN
La manifattura additiva (MA) di materiali metallici, comunemente chiamata stampa 3d, è un approccio inovativo alla produzione che permette di ottenere componenti metallici attraverso un processo manifatturiero che aggiunge il materiale livello dopo livello. Queste nuove tecnologie, ci permettono di produrre forme estremamente complesse, struttura più leggere e dei pattern superficiali che non sarebbe possibile ottenere con le tecniche manifatturiere tradizionali. La recente innovazione tecnologica, ha accelerato l'adozione di queste tecniche in ambito industriale, soprattutto per i processi di Electron Beam Melting e di Selective Laser Sintering. Questi processi, come vedremo, utilizzano una sorgente esterna di energia concentrata per fondere selettivamente della polvere metallica, livello dopo livello. In questa tesi, descriverò in dettaglio i processi Powder Bed Fusion, fornendo un quadro dei principi operazionali di base, le sfide, le criticità e i vantaggi rispetto ad altri processi manifatturieri tradizionali, i difetti tipici che possono verificarsi in questi processi, e come individuarli con le nuove tecniche di controllo della qualità. In particolare, parlerò degli hot-spot, delle aree circoscritte caratterizzate da un'elevata temperatura e da un processo di raffreddamento anomalo, spesso relativi a parti circondate da polvere non fusa, come angoli acuti o spigoli. Nella MA, questi difetti sono particolarmente importanti, in quanto la qualità dei componenti fabbricati è influenzata dal comportamento termico durante la fase di stampa. Infatti, gli hot-spot possono portare a delle modifiche della microstruttura del materiale solidificato, alla formazione di porosità e a difetti geometrici. Quindi, è esseziale analizzare e capire la storia termica del pezzo stampato, soprattutto durante la fase di certificazione del componente, nel quale vengono testate le qualità meccaniche del componente. In questa tesi, proporrò anche un algoritmo per la clusterizzazione delle aree del piatto di stampa caratterizzate da una storia termica simile utilizzando dei dati funzionali. Proporrò anche una implementazione dell'algoritmo in Python con uno studio su dati simulati, ponendo particolare enfasi sulle prestazioni del codice e sulla possibile implementazione in sistemi embedded.
\\[0.5cm]
\textbf{Parole chiave:} manifattura additiva, powder bed fusion, controllo qualità, hot-spot, machine learning, analisi di dati funzionali % Keywords (italian)