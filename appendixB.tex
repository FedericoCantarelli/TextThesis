\textcolor{red}{Implementazione iniziale dell'algoritmo. Next steps:
\begin{itemize}
    \item Parallelizzazione del bootstrap
    \item Struttura più organizzata
    \item Refactoring e commenti
\end{itemize}}
In this appendix, I want to suggest a potential implementation of the Bagging Voronoi clustering algorithm discussed in \emph{Chapter ~\ref{ch:baggingvoronoi}}. I decided to implement the algorithm in Python because it's currently the most widely used general-purpose programming language among developer \cite{lionel_sujay_vailshery_most_2023}. Furthermore, being a high-level and straightforward programming language, Python is easily comprehensible even to those unfamiliar with coding. If there is the need to implement the algorithm for real-time in-situ monitoring, it would be essential to rewrite the algorithm in C, as it offers significantly greater computational speed compared to Python. However, there isn't any existing library for functional data analysis already implemented in C, but in C++, which has comparable performances to C. I chose to construct a class structure to allow for easy integration into a Python package. The development of a package for this analysis in Python could be a future development of this thesis. Moreover, concurrent Python programming can be seamlessly integrated into cloud environments (such as Google Cloud Platform, Amazon AWS, or Microsoft Azure), leveraging also containerization and autoscaling feature available with those service providers in order to get faster results. To perform clustering I used k-means clustering, to compute the distance i used the Euclidean distance and as a clustering matching method the permutation of contincengy matrix explained in Section \ref{sec:bvc}.
\section{Python Code}
%\lstinputlisting[language=Python]{analysis-to-latex.py}

\section{Code Performances}
\textcolor{red}{Se riesco a fare la simulazione che funziona.}