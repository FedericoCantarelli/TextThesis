In this appendix, I want to suggest a potential implementation of the Bagging Voronoi clustering algorithm discussed in \ref{ch:baggingvoronoi}. I decided to implement the algorithm in Python due to its straightforward nature and the ability to execute code concurrently, leveraging on the parallel computing capacities that today's computers offer. If there is the need to implement the algorithm for real-time in-situ monitoring, it would be essential to rewrite the algorithm in C, as it offers significantly greater computational speed compared to Python. However, there isn't any existing library for functional data analysis already implemented in C, but in C++, which has comparable performances to C. The provided code represents an initial implementation of the function: it operates correctly, but it doesn't fully exploit the parallelization potential inherent in the bootstrapping approach. I chose to construct a class structure to allow for easy integration into a Python package. The development of a package for this analysis in Python could be a subsequent step. Moreover, concurrent Python programming can be seamlessly integrated into cloud environments (such as Google Cloud Platform, Amazon AWS, or Microsoft Azure), leveraging also containerization. With proper economic consideration and a well-thought-out business plan, this could potentially be marketed as a Software as a Service (SaaS) to companies that use PBF to print parts using the technologies detailed in this thesis. To perform clustering I used k-means clustering, to compute the distance i used the Euclidean distance and as a clustering matching method the permutation of contincengy matrix explained in Section \ref{sec:bvc}.
\lstinputlisting[language=Python]{analysis-to-latex.py}