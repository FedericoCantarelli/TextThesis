% ABSTRACT IN ENGLISH
Metal Additive Manufacturing (AM) techniques, often called 3D printing, are innovative production processes for crafting metal components through incremental material deposition, layer after layer. These technologies allow the production of complex shapes, lightweight structures, and surface patterns that can not be achieved with traditional methods. Recent technology development has accelerated the industrial adoption of metal AM procedures, particularly Electron Beam Melting and Selective Laser Sintering processes involving selective melting of metal powders using an external energy source. This thesis undertakes a comprehensive appraisal of powder bed fusion technologies, providing an in-depth evaluation of operational principles, challenges, benefits, and critical issues compared with other processes, the typical defects, and approaches to detect them. The main topic of this thesis is hot spot defect detection. Hot spots are localized areas exhibiting elevated temperatures for a sustained time and slower cooling drift, often attributable to their proximity to adjacent loose powder. In metal AM, temperature assumes a particular significance as most defects are due to anomalous thermal conditions. Moreover, a component's thermal history significantly influences the finished product's characteristics, including porosity, microstructure, and geometry. Thus, it is essential to correctly understand and analyze the thermal history of the printed part, especially during the testing and certification phase. The thesis also presents an algorithm for clustering areas on the build plate that share analogous thermal histories through functional data analysis, and it proposes an implementation of the algorithm in Python.
\\[0.5cm]
\textbf{Keywords:} metal additive manufacturing, powder bed fusion, defect detection, hot-spot, machine learning, functional data analysis

% 1782 caratteri
