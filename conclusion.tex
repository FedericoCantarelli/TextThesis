\begin{minipage}{0.7\textwidth}
\small
			\textit{We can imagine that this complicated array of moving things which constitutes “the world” is something like a great chess game being played by the gods, and we are observers of the game. We do not know what the rules of the game are; all we are allowed to do is to watch the playing. Of course, if we watch long enough, we may eventually catch on to a few of the rules.
            \\[1.5ex]
            -Richard Feynman}
\end{minipage}
\\[2.5ex]
\textcolor{red}{\textbf{To be completed.}}
%This chapter is the end of our journey into additive manufacturing for metallic components and functional data world. In \emph{Chapter ~\ref{ch:baggingvoronoi}} were just some demonstrations of how the classification algorithm could be employed with functional data to detect hot spots. Fundamentally, the strength of the algorithm lies in its ability to identify spatial dependencies and evolution in time or space, since in additive manufacturing time and space along z-axis are strongly linked. Such functional data afford us a deep understanding of the events unfolding in space/time, also allowing us to understand how the printing process evolve in time. Finally, the algorithm is highly versatile, allowing us to select the size of the lattice site and, consequently, the "magnification lens" through which we wish to approach the analysis. Indeed, we can use given its bootstrapping approach, it also fully leverages the capabilities of today's multi-core processors. But not just temperature profiles: functional data benefits from all the properties functions dos. Other examples of functional data could include electrons emitted during fusion in EBM processes over time, or the concentration of inert gas flowing through the printing chamber. The purpose of this thesis, in addition to providing an overview of AM processes for metals, was to reflect on how a relatively young branch of mathematics (late 90s) can contribute to the understanding of physical phenomena in AM.