% ABSTRACT IN ITALIAN
La manifattura additiva (MA) di materiali metallici, è un approccio inovativo alla produzione che permette di ottenere componenti metallici attraverso la fusione selettiva di materiali metallici. Queste nuove tecnologie, ci permettono di produrre forme estremamente complesse che non sarebbe possibile ottenere con le tecniche manifatturiere tradizionali. La recente innovazione tecnologica, ha accelerato l'adozione di queste tecniche in ambito industriale, soprattutto per i processi di Electron Beam Melting e di Selective Laser Sintering. Questi processi utilizzano una sorgente esterna di energia concentrata per la fusione selettiva della polvere metallica, livello dopo livello. Questa tesi descrive in dettaglio i processi Powder Bed Fusion, e fornisce un quadro dei principi operazionali di base, le sfide, le criticità e i vantaggi rispetto ad altri processi manifatturieri, i difetti tipici che possono verificarsi, e i diversi approcci per individuarli. La tesi è incentrata sugli hot-spot, delle aree circoscritte caratterizzate da un'elevata temperatura sostenuta nel tempo e da un processo di raffreddamento anomalo, spesso relativi a parti circondate da polvere non fusa. Nella MA metallica, la temperatura assume un ruolo critico, in quanto la qualità dei componenti fabbricati è influenzata dal comportamento termico durante la fase di stampa. Infatti, gli hot-spot possono portare a delle modifiche della microstruttura del materiale solidificato, alla formazione di porosità e a difetti geometrici che minerebbero le qualità meccaniche del pezzo. Questa tesi presenta anche un algoritmo per la clusterizzazione delle aree del piatto di stampa caratterizzate da una storia termica simile utilizzando i dati funzionali termici. Infine. la tesi propone un'implementazione dell'algoritmo in Python.
\\[0.5cm]
\textbf{Parole chiave:} manifattura additiva, powder bed fusion, controllo qualità, hot-spot, machine learning, analisi di dati funzionali % Keywords (italian)

% 1814 caratteri 254 parole