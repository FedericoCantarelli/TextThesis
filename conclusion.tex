\begin{minipage}{0.7\textwidth}
\small
			\textit{We can imagine that this complicated array of moving things which constitutes “the world” is something like a great chess game being played by the gods, and we are observers of the game. We do not know what the rules of the game are; all we are allowed to do is to watch the playing. Of course, if we watch long enough, we may eventually catch on to a few of the rules.
            \\[1.3ex]
            -Richard Feynman}
\end{minipage}
\\[1cm]
This chapter concludes our journey into metal AM and machine learning algorithms for HS defect detection. Regarding the systematic literature review, this thesis introduced AM in general in Chapter \ref{ch:Introduction}, while in Chapter \ref{ch:Metal_AM} it described in detail the principles of operation, materials used, and physical phenomena underlying SLS and EBM processes. In Chapter \ref{ch:defects} and Chapter \ref{ch:state_ot_the_art}, the thesis minutely described the techniques used for process monitoring, common sensors employed, and machine learning-based approaches for defect detection of HSs. The research approach used was structured and detailed in Appendix \ref{ap:research}. Thus, this thesis achieved its primary purpose.
On the other hand, regarding the innovative approach proposed in Chapter \ref{ch:baggingvoronoi}, experiments have shown that its real-time application using a single-core Python script is not feasible. Next step could be to test a multi-core implementation written in more computationally efficient languages. In addition, Chapter \ref{ch:baggingvoronoi} presented how the algorithm can be applied to simulated data for clustering areas of the plate that exhibit similar thermal profiles correlated in space and time, succeeding in creating a temperature heatmap. Furthermore, with the proposed approach, functional data could be leveraged during certification phase to decide whether or not the part is usable in the context for which it was designed. We can draw a final conclusion on the proposed innovative approach by referring to the desirable features for a real-time defect detection system described in \cite{yan_real-time_2021} (2021).
According to the authors, an in-situ defect detection method must be able to address the following challenges.
\begin{enumerate}
    \item \textit{High dimensionality.} High-resolution images are comprised of millions of pixels. This point is completely satisfied. FPCA and the initial selection of the number of reduce significantly problem's dimensionality. Even more important, it can be reduced to meet the problem domain, since these parameters can be tuned.
    \item \textit{High velocity.} High-speed cameras may acquire thousands of frames per second, which requires a computationally efficient real-time analysis of image frames. More testing needs to be done before this requirement can be met. The need to perform multiple bootstraps causes the time required to cluster the single frame to increase, and this computation time could be reduced using multi-core programming. However, this is not currently possible.
    \item \textit{No anomaly labels.} In most industrial applications, few anomaly samples are available, and normally, no labels are provided to assess whether the sample is anomalous. This point is also satisfied. The proposed algorithm is unsupervised, and by using the automatic anomaly identification technique described earlier, it would be able to raise alarms in the case of OOCs.
    \item \textit{Complex spatio-temporal correlation structure.} Neighbor pixels are spatially correlated, and consecutive image frames are temporally correlated. As we saw in the data simulation section, the algorithm can correctly identify isolated areas of the plane by exploiting the Voronoi tessellation that defines the neighborhood. In addition, we can select weights to compute representative functions so that any anisotropies in the thermal profiles can also be considered. On the other hand, functional data captures the time correlation structure.
    \item \textit{Measurement uncertainty.} Measurement noise may mask relevant spatial and temporal patterns. In the case of functional data, this problem does not exist since it is not the individual observation that is anomalous but the function itself that generated it. Moreover, we can easily impute missing values with the FDA once the appropriate functional basis is chosen.
\end{enumerate}
Finally, I think the proposed method has three significant advantages: (i) the algorithm is highly versatile, allowing us to choose the "magnification lens" through which we wish to approach the analysis, (ii) once the lattice sites have been classified we have a measure of the statistical certainty with which those observations belong to that particular cluster, and (iii) by exploiting the functional data we have full knowledge of the phenomenon, collecting valuable insigths that can be used also in models simulation.
